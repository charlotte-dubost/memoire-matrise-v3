%% -----------------------------------
%% ---> À MODIFIER PAR L'ETUDIANT / TO BE MODIFIED BY THE STUDENT <---
%% -----------------------------------
%%
%% Commandes qui affichent le titre du document, le nom de l'auteur, etc.
\newcommand\monTitre{Roue Cyr en matériaux composites : enrichir le langage acrobatique et scénique d'une discipline de cirque}
\newcommand\monPrenom{Charlotte}
\newcommand\monNom{Dubost}
\newcommand\monDepartement{génie mécanique}  % Department
\newcommand\maDiscipline{Génie mécanique}
\newcommand\monDiplome{M}        % (M)aîtrise ou (D)octorat / (M)aster or Ph(D)
\newcommand\anneeDepot{2020}    % Year
\newcommand\moisDepot{Juillet}       % Month
\newcommand\monSexe{F}           % "M" ou "F" = Gender
\newcommand\PageGarde{N}         % "O" ou "N" = Yes or No
\newcommand\AnnexesPresentes{O}  % "O" ou "N". Indique si le document comprend des annexes. / If the thesis includes annexes = O or N = No.
\newcommand\mesMotsClef{Cirque,roue,cyr,composites,elasticité}
\usepackage{gensymb}
%%
%%  DEFINITION DU / OF JURY
%%
%%  Pour la définition du jury, les macros suivantes sont definies:
%%  \PresidentJury, \DirecteurRecherche, \CoDirecteurRecherche, \MembreJury, \MembreExterneJury
%%
%%  Toutes les macros prennent 3 paramètres: Sexe (M/F), Nom, Prénom
%%  All the macros have 3 parameters: Sex (M/F), Last name, First name
\newcommand\monJury{\PresidentJury{F}{Nom}{Prenom}\\
\DirecteurRecherche{M}{Gosselin}{Frédérick}\\
\CoDirecteurRecherche{F}{Ross}{Annie}\\
\CoDirecteurRecherche{M}{Therriault}{Daniel}\\
\MembreJury{M}{Nom}{Prénom}\\
\MembreExterneJury{M}{Nom}{Prénom}}

