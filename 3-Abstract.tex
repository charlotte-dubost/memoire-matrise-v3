% Abstract
%
% Résumé de la recherche écrit en anglais sans être
% une traduction mot à mot du résumé écrit en français.

\chapter*{ABSTRACT}\thispagestyle{headings}
\addcontentsline{toc}{compteur}{ABSTRACT}
%

A Cyr wheel is a circus apparatus consisting of a human sized metallic ring. The user stands inside the wheel and performs acrobatic figures while spinning. The weight, rigidity and geometry of the wheel are thus determinant parameters for the user’s motion. Since it has been invented, the Cyr wheel has been revisited by circus artists and manufacturers to create hybrid apparatuses. As composite manufacturing for sports spreads to circus, Cyr wheel adepts wonder what a lighter, more flexible wheel would result in. These wheels are composed of curved beams connected by sleeves: this implies fabrication costs and excludes the ‘trial and error’ method to answer the question. Thus, no composite Cyr wheel project has been completed yet. 

To this day there isn’t any scientific publication specific to Cyr wheel. Nevertheless, its dynamics relate to Euler’s disk. We will adapt theoretical models in the literature as a basis to determine how the Cyr wheel geometry influences dynamic stability. We will also focus on the jump of a wheel being bent and released, and thus refer to Yang and Kim article about energy storage in elastic rings as a comparative study.

The design, manufacturing and test of a 3D printed composite Cyr Wheel prototype combines practical experience of the apparatus, familiarity with composite materials, large-scale 3D printing processes and dynamics of Euler’s disk-like motions. The outcome will shed light on the interests for manufacturers to invest in composite Cyr wheels. 
Our objective is to determine which mechanical properties and geometry will optimize the stability and jumps of a Cyr wheel, 3D print a prototype and test acrobatic possibilities with circus artists.
\\
