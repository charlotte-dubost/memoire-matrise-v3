% Résumé du mémoire.
%
\chapter*{RÉSUMÉ}\thispagestyle{headings}
\addcontentsline{toc}{compteur}{RÉSUMÉ}

La roue Cyr, agrès de cirque, est un anneau métallique à taille humaine. L’utilisateur s’y agrippe et exécute des figures acrobatiques en tournant. Le poids, la rigidité et la flexibilité de la roue sont donc des paramètres déterminants pour le mouvement de l’utilisateur. Depuis son invention, la roue Cyr a été revisitée par les artistes et les fabricants pour créer des agrès hybrides. Alors que le cirque s’approprie les matériaux composites utilisés pour l’équipement sportif, les adeptes de la roue Cyr se demandent ce qu’une roue plus légère et flexible apporterait à la discipline. Ces roues sont constituées de poutres courbes connectées par des manchons, ce qui implique des coûts de fabrication excluant la méthode ‘essai-erreur’ pour répondre à la question. Tous les projets alliant roue Cyr et matériaux composites restent ainsi inachevés. 

A ce jour la roue Cyr ne fait l’objet d’aucune publication scientifique. Cependant, son mouvement est similaire à celui du disque d’Euler : nous adapterons donc les modèles théoriques issus de la littérature comme base pour déterminer comment la géométrie de la roue Cyr influence sa stabilité dynamique. Nous étudions aussi le saut d’une roue comprimée puis relâchée, en référant aux travaux de Yang et Kim sur le stockage d’énergie élastique dans des anneaux.

La conception, la fabrication et les tests d’une roue Cyr en matériaux composites combine une expérience pratique de l’agrès, l’impression 3D de matériaux composites à grande échelle et la dynamique du disque d’Euler. L’issue de ces travaux contribuera à clarifier l’intérêt, pour les fabricants, d’investir dans les roues Cyr en composites.
Notre objectif est de déterminer les propriétés du matériau et la géométrie optimales pour la stabilité et le saut d’une roue Cyr, d’imprimer un prototype et de tester son potentiel acrobatique avec des artistes de cirque.
